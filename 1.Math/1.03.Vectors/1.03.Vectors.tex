\documentclass{article}

% Language setting
% Replace `english' with e.g. `spanish' to change the document language
\usepackage[english,russian]{babel}

% Set page size and margins
% Replace `letterpaper' with `a4paper' for UK/EU standard size
\usepackage[letterpaper,top=2cm,bottom=2cm,left=3cm,right=3cm,marginparwidth=1.75cm]{geometry}

% Useful packages
\usepackage{amsmath}
\usepackage{graphicx}
\usepackage[colorlinks=true, allcolors=blue]{hyperref}

\title{ДЗ-Векторы}

\begin{document}
\maketitle

\section{Базис}

Базис – набор линейно независимых векторов, которыми можно выразить любой вектор в пространстве.

Пример 1:\\
$e_{1}=[1,0]$\\
$e_{2}=[0,1]$\\

Пример 2:\\
$e_{1}=[1,0,0]$\\
$e_{2}=[0,1,0]$\\
$e_{3}=[0,0,1]$\\

\section{Разложение вектора [20,45,34] по базису}

Базис:\\
$e_{1}=[1,0,0]$ \\
$e_{2}=[0,1,0]$ \\
$e_{3}=[0,0,1]$ \\

$[20,45,34] = 20*e_{1}+45*e_{2}+34*e_{3}$
\subsection{Евклидово и манхеттенское расстояние}

Евклидово расстояние – кратчайшее (минимальное) расстояние между двумя точками «по прямой, как ворона летит».\\

Манхеттенское расстояние – кратчайшее расстояние между двумя точками, если идти “прямыми углами” по координатным сеткам.\\

Евклидово расстояние – единственное между двумя точками, манхеттенских расстояний может быть несколько.\\

a=[23,-34,56],b=[45,67,-28]

\paragraph{Евклидиво расстояние:}

$D(a,b) = \sqrt{ (23-45)^2 + (-34-67)^2 + (56+28)^2} = \sqrt{484+10201+7056} = 133,195345264$

\paragraph{Манхеттенское расстояние }

D(a,b)= |23-45| + |-34-67| + |56+28| = 22+101+84=207


\section{Метрическое пространство }

 Пусть есть множество X, для которого есть некая функция d(a,b), которая позволяет вычислить расстояние между любыми элементами множества. Тогда метрическое пространство - пара (X,d), а функция d(a,b) - метрика.\par

Метрика должна обладать следующими свойствами:
\begin{itemize}
    \item неотрицательность, т.е. для каждого a и b, принадлежащих X, d(a,b) >= 0. d(a,b)=0 тогда и только тогда, когда a=b.
    \item симметрия, т.е. для каждого a и b, принадлежащих X, d(a,b)=d(b,a)
    \item неравенство треугольника, т.е. для каждых a, b, c, принадлежащих X,: d(a,c) <= d(a,b)+d(b,c)
\end{itemize}


\end{document}